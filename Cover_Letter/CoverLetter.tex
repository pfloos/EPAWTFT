\documentclass[10pt]{letter}
\usepackage{UPS_letterhead,xcolor,mhchem,mathpazo,ragged2e,url}
\newcommand{\alert}[1]{\textcolor{red}{#1}}
\definecolor{darkgreen}{HTML}{009900}


\begin{document}

\begin{letter}%
{To the Editors of the Journal of Physics: Condensed Matter (JPCM)}

\opening{Dear Editors,}

\justifying
Following the recent invitation of Ania Wronski, please find enclosed our manuscript entitled \textit{``Perturbation Theory in the Complex Plane: Exceptional Points and Where to Find Them''}, which we would like you to consider as a Topical Review in \textit{J. Phys. Cond. Mat.}

This multidisciplinary review explores the non-Hermitian extension of computational quantum chemistry to the complex plane and its link with perturbation theory.
In particular, we focus on its mathematical roots and connections with physical phenomena such as quantum phase transitions and exceptional points in the complex plane.
We begin by presenting the fundamental concepts behind non-Hermitian extensions of quantum chemistry into the complex plane, including the Hartree--Fock approximation and
Rayleigh--Schr\"odinger perturbation theory. 
We then provide a comprehensive review of the various research that has been performed around the physics of complex singularities in perturbation theory, with a particular focus on M{\o}ller--Plesset theory.
Finally, several resummation techniques are discussedthat can improve energy estimates for both convergent and divergent series, including Pad\'e and quadratic approximants.
Throughout this review, we present illustrative and pedagogical examples based on the ubiquitous Hubbard dimer at half-filling, reinforcing the amazing versatility of this powerful simplistic model.

Due to the genuine interdisciplinary nature of the present article and its pedagogical aspect, we believe that it will be of interest to a wide audience within the physics and chemistry communities.
We hope that the editors and the reviewers of \textit{JPCM} will find this topical review enjoyable and educative.

We suggest Paola Gori-Giorgi, Jeppe Olsen, So Hirata, Peter Knowles, and Kieron Burke as potential referees.
We look forward to hearing from you soon.

\closing{Sincerely, the authors.}


\end{letter}
\end{document}






