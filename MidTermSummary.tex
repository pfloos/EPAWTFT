\documentclass{article}
\usepackage[T1]{fontenc}
\usepackage[english]{babel}
\usepackage{float}
\usepackage{hyperref}
\usepackage{siunitx}
\usepackage{graphicx}
\usepackage[top=3cm,bottom=5cm,left=3cm,right=3cm]{geometry}
\usepackage{caption}
\usepackage[version=4]{mhchem}

\title{Internship Summary: Mid-Term Review}

\author{Antoine Marie}

\begin{document}

\maketitle

\section{Generalities}

\subsection{Objectives}
Our goal is to understand the physics of exceptional points (EPs) and perturbation series, i.e., where are EPs localize in the complex plane and how do they affect the convergence properties of the perturbation series. To do this we use the electronic ground-state of the spherium model (i.e., two opposite-spin electrons restricted to remain on a surface of a sphere of radius $R$) as a playground. 

\subsection{Variables}

There are many variables that influence the physics of EPs and we must rationalize how each variable affects the location of these EPs.
\begin{itemize}
	\item Partitioning of the Hamiltonian and the actual zeroth-order reference: weak correlation reference [restricted Hartree-Fock (RHF) or unrestricted Hartree-Fock (UHF) references, M{\o}ller-Plesset (MP) or Epstein-Nesbet (EN) partitioning], or strongly correlated reference.
	\item Basis set: minimal basis or infinite (i.e., complete) basis made of localized or delocalized basis functions
	\item Radius of the spherium that ultimately dictates the correlation regime, i.e., weak correlation regime at small $R$ where the kinetic energy dominates, or strong correlation regime where the electron repulsion term drives the physics.
\end{itemize}

\subsection{``Classification'' of the bibliography}

We can categorize what have been done already in 3 groups:

\begin{itemize}
\item Gill, Handy, Nobes, \ldots in the late 80's noticed that there are links between deceptive/erratic convergence of the MP pertubation series and spin contamination of the wavefunction.

\item Chaudhuri, Olsen, \ldots in the late 90's highlighted links between singularities in the complex plane and convergence/divergence schemes of the MP series with a two-state model.

\item Sergeev, Goodson, Stillinger in the 2000's studied this problem from a more mathematical point of view: $\alpha$/$\beta$ classification of the singularities, Pad\'e approximant, critical point on the real axis, etc.
\end{itemize}

\section{Results}

\subsection{Spin contamination}
The UHF description of \ce{H2} and spherium is analog. 
Gill's UHF description of \ce{H2} in the minimal basis (1988) shows that the spin contamination of the doubly excited state RHF by triplet RHF wavefunctions are link to the poor convergence of the UMP series. 
For spherium in the minimal basis the RHF reference is bloc diagonal but the UHF reference has non-zero matrix elements between triplet states and $p_z^2$ singlet. 
The matrix elements are

\begin{equation}
	\lambda\frac{\sqrt{-3+2R}(25+2R)\sqrt[2]{75+62R}}{280 R^{3}}
\end{equation}
% you should try to simplify this expression 

The matrix elements are real for $R > 3/2$ and $R < -75/62$. 
This is coherent with the Coulson-Fisher points of spherium (Burton 2019).
However, Gill does not talk about the singularity structure of $E(z)$!

\subsection{Two state model}

Olsen et al.~developped a two-state model which allowed them to rationalize many schemes of convergence. 
Can we do this with our perturbation series? 

\begin{itemize}
	\item RHF: The schemes of convergence predicted by the two-state model fit with our plots of the coefficients for both MP and EN partitioning.
	\item UHF: The schemes predicted by the two-state model are not correct due to its intrinsic limitations. 
\end{itemize}
In the UHF framework we must consider more than two states because of the spin contamination $\rightarrow$ this is expected that the two state model works for RHF and not for UHF.

\subsection{Singularity structure}

Sergeev 2005:
\begin{quote}
\textit{``Class $\alpha$ singularities are isolated complex conjugate pairs of square-root branch points, which correspond to the avoided crossings between the ground state and an excited state on a path along the real axis, while class $\beta$ singularities lie on or near the real axis and correspond to critical points at which one or more electrons dissociate from the nuclei.''}
\end{quote}

\subsubsection{Class $\beta$}

According to this classification we could predict that there will be only class $\alpha$ singularities for spherium because the electrons are restricted to the surface of the sphere.
Nonetheless, in the UHF framework we obtain class $\beta$ singularities. 
How is this possible?

We can prove that there are a value of $z$ at which the electrons form a cluster and dissociate from the nuclei (Stillinger 2000 and Sergeev 2005). 
By analogy with thermodynamics (Baker 1971) this value of $z$ is a critical point for $E(z)$. 
At this critical point the energy undergoes a phase transition from a discrete spectrum to a continuum of ionization states. 
So it is possible that class $\beta$ singularities are not just linked to ionization but more generally to phase transitions and symmetry breaking.
%T2: very true indeed.
In our UHF framework we can see the apparition of those class $\beta$ singularities appear at the Coulson-Fischer points within the MP partitioning.

\subsubsection{Class $\alpha$}

We obtain as expected the class $\alpha$ singularities characteristic of an avoided crossing between the ground state and a doubly excited state. 
But they are sometimes in the positive half plane, sometimes in the negative half plane and they can also be on the pure imaginary axis.
Guess: this is due to the value/sign of $E_1$. 
How can we prove this ?

\subsection{Strongly correlated}

We can use the ``Laguerre'' basis to define an adiabatic connection with a ``localised'' reference. 
We find that the radius of convergence is greater than unity for all $R$ in the minimal and the $n = 3$ basis. 

\section{Open questions}

\begin{itemize}
	\item Guess: adding diffuse functions as in the case of ionization $\beta$ singularities is not necessary because the symmetry breaking is already well described in the minimal basis. 
	Can we prove this?
	\item Effect of an increase of the size of the basis set for the strongly correlated reference?
	\item Rationalize very precisely the singularity structure of the MP partitioning: avoided crossing (sharp or not) between which states, symmetry breaking, \ldots
	\item Study the symmetry breaking in the EN partitioning
	\item Why the radius of convergence for the strongly correlated partitioning is always greater than 1? 
	Is there a feature of the basis functions that can make a series divergent?
\end{itemize}

Antoine is happy :-)

\end{document}

